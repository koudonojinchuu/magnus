%% This is AMSLaTeX source, included by guidelines_incl.tex.


%%--------------------------------------------------------------------
\section{A {\bf cvs} Tutorial}

%%....................................................................
\subsection{Introduction}


The \magnus\ project uses the Concurrent Versions System {\bf cvs(1)}
for version control.

This tutorial is not a substitute for the {\bf man} pages. Rather, it
is an overall explanation of the {\bf cvs} commands, and a discussion
of the very idea of version control.

{\bf cvs} keeps a unique directory tree, called the {\em master
repository}, which contains {\bf rcs(1)} archives of all project
files. Each developer keeps at least one {\em working directory}, in
which to perform the usual edit-compile-link-run-crash cycles. The
directory structure of a working directory usually contains a
subforest of the master tree.

The {\bf cvs} paradigm differs from those with which you may be
familiar, as it permits concurrent editing of files. That is, it does
not lock a given source file so that only one developer at a time can
edit it.

A locking mechanism attempts to prevent conflicting changes to a
source file by two or more people {\em before the fact}, by announcing
to all developers that the file is `locked' by someone, and therefore
should not be edited by anyone else.  However, any such mechanism has
loopholes through which it can be inadvertently defeated. This can
result in one revision shadowing another with no warning.

In recognition of the fact that conflicts will occur, {\bf cvs}
automatically checks for them, and attempts to resolve them, {\em
after the fact}. This ensures that conflicts are always caught, but
places a greater burden on those developers who edit a file
concurrently. Because, as they attempt to check their changes into the
master, {\bf cvs} will run {\bf merge(1)} on the original revision and
the two changed ones. If the changes do not overlap, they are merged
automatically, which is presumably the right thing.  Otherwise, one of
the developers will have to resolve the conflict by hand, using the
output from {\bf merge}.


%%....................................................................
\subsection{Getting Started}

You must set the environment variable CVSROOT to the full path of the
master repository. For example, put the line

\vspace{8pt}

{\tt setenv CVSROOT /home/users/magnus\_repository}

\vspace{8pt}

\noindent in your {\tt .cshrc} file.

\vspace{8pt}

\begin{quotation}

The contents of \$CVSROOT are to be manipulated by {\bf cvs} only.

NO ONE, UNDER ANY CIRCUMSTANCES OF ANY KIND WHATSOEVER, IS TO DIRECTLY
ALTER OR REMOVE ANY FILE OR DIRECTORY IN OR BELOW $CVSROOT.

\end{quotation}


You must make a working directory. You do not necessarily need a copy
of the entire master, but this is the easiest to do at first.  Switch
to the directory where you want your working directory to go, then
invoke the command

\vspace{8pt}

{\tt cvs checkout magnus}

\vspace{8pt}

This command is decribed in more detail, below. Whatever you do,
remember that you must work in a directory tree created by the {\bf
checkout} command, since {\bf cvs} will not recognize a tree which it
did not create, unless you take special action (see the {\bf import}
command).


%%....................................................................
\subsection{Common Commands}

There are only five {\bf cvs} commands which a developer typically
needs. Perhaps it goes without saying, but one should not use other
commands, or the more esoteric options of the five below, until one
fully understands the consequences.

You will see that there is some subtle interaction between the {\bf
checkout}, {\bf add}, and {\bf update} commands, because {\bf cvs}
must be explicitly told which files in your working directory are to
be considered for inclusion in the master. This conveniently allows
you to have completely private files in your working directory,
without {\bf cvs} constantly trying to inappropriately check them into
the master.

Below, `{\em module}' means a file or directory path relative to the
master root, or a symbolic tag which refers to a collection of files.
And `{\em files}' means files or directories in the current directory.

\begin{description}

\item{{\bf cvs checkout} [options] {\em modules}...}

The purpose of this command is to make a working copy of some
subforest of the master, along with the administrative files which
{\bf cvs} needs to operate.

This is typically used to initialize your working directory, but you
can also use it to update an existing working subdirectory (see {\bf
cvs(1)}).

You may also, for example, use this command to create a duplicate of
some working subdirectory, in which to do experimental work without
disturbing the `real' one.

For this reason, the {\bf create} command does not automatically put
the checked out files in the same location, relative to the master
root, as they are in the master. Rather, it puts the roots of any
trees it creates in the directory from which it was invoked.

If any of the {\em modules} you specify are directories, {\bf
checkout} is recursive, unless you give the {\tt -l} (for local)
option. The {\tt -n} option prints a summary of what would happen if
the command were executed, but does not actually change anything. See
{\bf cvs(1)} for more options.



\item{{\bf cvs update} [options] {\em files}...}

This changes your working directory, not the master repository.

This brings your working files up to date with any changes which
others have made in the master since the last time you did {\bf
update} or {\bf checkout}. It is recursive unless you give the {\tt
-l} option. It {\em does not} check any working files you have changed
into the master. Use {\bf commit} for that.

Examples:

\begin{itemize}

\item
Someone has fixed a bug in {\tt
magnus/back\_end/general/include/List.h}, and now you want it.
Switch to the directory which contains your working root, and do

\vspace{8pt}

{\tt cvs update magnus/back\_end/general/include/List.h}

\vspace{8pt}

\item
As above, but you can't remember what file in {\tt general/} was
fixed.  If you do

\vspace{8pt}

{\tt cvs update magnus/back\_end/general}

\vspace{8pt}

\noindent then all of your files in {\tt general/} are brought up to date.

\vspace{8pt}

But what if you are working on, say, {\tt general/src/Set.C}, and you
are not ready to check it in? Not to worry. The {\bf update} command
never checks files in.

\vspace{8pt}

But what if {\em someone else} has also edited {\tt Set.C} in the mean
time? In that case, {\bf update} will try to merge the other changes
with your working copy of {\tt Set.C}. If you were working on disjoint
parts of {\tt Set.C}, the merge is automatic. If you were both editing
the same lines, {\bf update} will inform you that there is a conflict
which you must resolve by hand (see \S\ref{resolving_conflicts}).

\end{itemize}



\item{{\bf cvs add} [options] {\em files}...}

As you add new files to your working tree which are to be included in
the master, you must inform {\bf cvs} of them. Otherwise, {\bf cvs}
will assume that the file is for your private use, not part of the
project, and ignore it.

After you have made a new file, e.g., {\tt magnus/doc2/blurb.tex}, tell
{\bf cvs} that it is part of the project by switching to {\tt
magnus/doc2/}, and doing

\vspace{8pt}

{\tt cvs add blurb.tex}

\vspace{8pt}

Of course this assumes that the directory {\tt magnus/doc2/} was
either created by the {\bf checkout} command, or that {\bf cvs} was
told about it with the {\bf add} command.

The file `{\tt blurb.tex}' will not be checked into the master until
your run {\bf commit}.



\item{{\bf cvs remove} [options] {\em files}...}

Use this command to announce that the indicated files and directories
are no longer part of the project. The command is recursive unless you
give the {\tt -l} option.

Consider carefully whether it is for you to decide that a file no
longer part of the project.

The command requires that you first {\tt rm} the indicated files and
directories from your working tree. It does not change the master
until you run {\bf commit}.

Do {\em NOT\/} use this command when you merely wish to remove part of
your working tree, pending a {\bf checkout} to restore it. See the
{\bf release} command.



\item{{\bf cvs commit} [options] [{\em files}...]}

Use this command to check files into the master. They must all reside
in directories created with the {\bf checkout} command, or announced
with the {\bf add} command.

If you do not specify a file or directory, the current directory is
assumed. This command is recursive unless you give the {\tt -l}
option. It supports the {\tt -n} option, which prints a summary of
what would happen if you did commit, without actually changing
anything.

{\bf cvs} may detect that some of the specified files are not up to
date with the current revision in the master. That is, someone else
has also edited one of the files you want to commit, since the last
time you ran {\bf commit} or {\bf update}. In this case, {\bf cvs}
issues a warning to run {\bf update}, and exits without commiting any
files.

By default, this will invoke the editor pointed to by your EDITOR
environment variable, to ask for a log message. You can suppress this
with the {\tt -m ""} option. You may wish to alias this.


\end{description}



%%....................................................................
\subsection{Other Commands}


\begin{description}


\item{{\bf cvs diff} [options] [{\em files or directories}...]}



\item{{\bf cvs export} [options] [{\em files or directories}...]}

\item{{\bf cvs import} [options] [{\em files or directories}...]}

\item{{\bf cvs release} [options] [{\em files or directories}...]}

\item{{\bf cvs status} [options] [{\em files or directories}...]}


\end{description}



%%....................................................................
\subsection{Resolving Conflicts}\label{resolving_conflicts}
