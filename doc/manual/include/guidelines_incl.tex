%% This is AMSLaTeX source, included by guidelines.tex.


%%%%%%%%%%%%%%%%%%%%%%%%%%%%%%%%%%%%%%%%%%%%%%%%%%%%%%%%%%%%%%%%%%%%%%
\chapter{Project Guidelines}

%%--------------------------------------------------------------------
\section{Organisation of Code}

It is essential to bear in mind that your code will have to be read
and understood by other people, at first to be properly integrated into
the system, and possibly later to be modified, extended, or otherwise
adapted.

Therefore, one should strive to keep the implementation as modular as
possible with clearly defined interfaces. Beyond this general
principle, the following practices are important.

%%....................................................................
\subsection{Naming conventions}

Public identifiers should be as self-documenting as possible.  This is a
matter of taste and judgement: we considered \code{FPGroup} to be too
terse, writing \code{FinitelyPresentedGroup} instead, but this made
source lines too long to read.

Our spelling conventions for {\em public} identifiers are:

\begin{itemize}
\item
Type and class names start with upper case.
\item
Names of constants are in all capitals.
\item
Member and method names start with lower case, but significant subwords
start with upper case. No underscores.
\end{itemize}

Local identifiers need not be as descriptive as public ones, but
should still convey their meaning and use.

%%....................................................................
\subsection{Files and Banners}

The program source is broken into files \code{*.[hC]}, generally
containing the implementation of a single class, with possibly in
addition closely related local servant classes.

Each file starts with a banner, containing the following mandatory
parts: revision identification and number, brief description of the
contents, principal author, status, and history of implementation and
modifications with authors. Blanks are in 
\file{magnus/doc/templates/Blank.[Ch]}.

Additionally, you may want to add, according to the circumstances,
other entries such as literature references, testing report summary,
implementation notes, next implementation steps, and bugs (suspected
and known).

%%....................................................................
\subsection{Comments}

\begin{itemize}
\item
Comment each data member and method of a class, in the header file.
Say what it is used for or what it does. Detailed information which is
not of interest to the user of the class goes in the source file or
the reference manual.
\item
Comment your source code so that those with a sufficient background in
infinite group theory and C++ can understand it as though they wrote
it.
\item
Some comments are not explanatory, but are meant to flag lines which are
questionable, or are `to do' reminders, etc. Mark these with a comment
containing `@' followed by your initials. That way we know who wrote
it, and no one has to sift through all comments to find problems; they
can be \code{grep}-ed.

Do not use @-marked comments purely for traceability, e.g., to mark
code you've added and are sure of. Those comments belong in the
revision history.

When you alter code written by someone else, you can address your
comments on the change to that person with, e.g., {\tt @rn:@stc}.

\end{itemize}

%%....................................................................
\subsection{Writing Up the Documentation to Your Code}

Apart from the comments and general structure of your code, the
quality of the documentation you write is of crucial importance,
because to begin with your code will have to be integrated into the
\magnus\ system, and later it may become necessary or desirable to
modify or extend it or localise later bugs.

Roughly, your documentation for each part of your implementation
should be divided into two parts: a general part and a reference part.

%% . . . . . . . . . . . . . . . . . . . . . . . . . . . . . . . . . .
\subsubsection{The General Part}

Here you discuss what other implementers need and want to know about
your code in order to use it, without necessarily understanding its
inner workings.

\begin{quote}
The worst possible documentation to create for an object-oriented
system is a stand-alone description of the semantics of each method on
a class-by-class basis. This approach tends to generate a great deal
of useless documentation that no one reads or trusts, and fails to
document the more important architectural issues that transcend
individual classes, namely, the collaborations among classes and
objects. It is far better to document these higher-level structures,
which can be expressed in diagrams of the notation but have no direct
linguistic expression in the programming language, and then refer
developers to the interfaces of certain important classes for tactical
details.
\footnote{Grady Booch, ``Object-Oriented Analysis and Design'', 2nd ed.}
\end{quote}

Thus you might give

	\begin{itemize}
	\item
		A description of the public classes (those intended for
		further use) and their features (such as for instance
		special efficiency and memory considerations);
	\item
		Important parts of the public class interfaces.
	\item
        Diagrams of derivation and owns-uses relationships among
        classes.
	\end{itemize}

Booch diagrams may be appropriate for the last. There is a simple
example in \S\ref{user_classes}; several others are in the reference
manual. Here is a brief list of the notation:

\vspace{20pt}

\epsfbox{ps/booch.ps}

\vspace{20pt}


%% . . . . . . . . . . . . . . . . . . . . . . . . . . . . . . . . . .
\subsubsection{The Reference Part}

Here you discuss the inner workings of your code, i.e. what people
need and want to know to understand and possibly modify or extend your
code in any way.  This may include:

	\begin{itemize}
	\item
		implementation ideas, principles, and techniques;
	\item
		algorithms;
	\item
		altenatives to implementation decisions;
	\item
		possible extensions;
	\item
		any other information as appropriate.
	\end{itemize}


%%--------------------------------------------------------------------
\section{Special Restrictions}

This section explains miscellaneous requirements and restrictions of
the system.

\begin{itemize}
\item
The final version of your code must {\em never} write to \code{cout}.
The session manager uses \code{cout} for communication with the front
end, and would be confused by any unofficial chatter.  This
restriction does not apply to test programs.

\item
Your debugging code, compiled conditionally on \code{DEBUG}, may write
to \code{cerr}, but all error messages and warnings intended for the
production version should be passed to the global functions
\code{error(char*)} and \code{warn(char*)}, resp.
\end{itemize}


%%--------------------------------------------------------------------
\section{Error Checking}

%%....................................................................
\subsection{Levels of Error Checking: \code{SAFETY} switch}

There are currently 3 levels of run-time error checking which can be
switched with the compiler flag \code{SAFETY}, defined in
\file{magnus/back\_end/global/global.h}.

The values of \code{SAFETY} currently defined mean:

\begin{enumerate}
\setcounter{enumi}{0}

\item
Maximal efficiency and speed, no error checking at all;
`garbage in --- garbage out'.

\item
Simple, inexpensive (fast) error checking, such as array bounds.

\item
More systematic and thorough error checking, tending among things to also
capture semantic errors.

\end{enumerate}

We do not expect level 2 to catch all possible errors.

Trap class interface errors quietly wherever reasonable. For example,
if an integer parameter \code{size} which suggests an initial hash table
size should be of the form $2^{n-1}$, but is not, just use the least such
integer greater than \code{size}.

%%....................................................................
\subsection{Dealing with Errors}


\magnus\ will doubtless migrate to exception generation and
handling. As there are no widely available compilers with these
features yet, do the following: Report fatal errors by calling the
global function \code{error(char*)} with an appropriate message. This
function will print the message and terminate the program.
If some condition is probably an error on the caller's part,
but can be safely overlooked (at least locally), you may call the global
function \code{warn(char*)}, which does not terminate the program,
instead of \code{error}.

%%....................................................................
\subsection{\code{DEBUG} toggle}

Whereas the \code{SAFETY} switch permits safer albeit slower execution,
the idea of the \code{DEBUG} toggle is to provide somewhat systematic
features to set traps, monitor the progression of execution and
localise semantic as well as coding errors in algorithms.
% @rn See the discussion in \S\ref{low_level_tests}.


%%--------------------------------------------------------------------
\section{Testing Procedures}

This section describes the testing procedures for \magnus.
Roughly, one may distinguish two levels of tests.

\begin{itemize}
\item
General `external' tests, which test the behaviour
of the system or its components
with respect to their external calling interfaces, viz.~for classes
their public methods.

\item
Specific low-level tests, which may be more specialised in their
way of verifying the internal functioning of (functions of) the
system. For instance, if a \code{SetOf} class constructs hash tables,
the implementer may desire or need to devise specific tests which
verify the internal layout of the tables.

\end{itemize}

The external tests (with as the need arises their associated test data
sets) are mandatory, as these are necessary to know whether the system
functions at all. Low level tests are a function of their necessity in
the course of implementation and not an end in themselves. For these, the 
guiding principle is to write them only as need dictates, for instance
when debugging; but if writing some, then to keep these for possible
reuse.


The aim is to 

\begin{itemize}
\item
build up a set of tests for \magnus\ concurrently
to the development of the system;

\item
organise this set as a reusable library of standard tests for
the system.

\end{itemize}

The reasons are the obvious:

\begin{itemize}

\item
Tests are onerous to devise and implement;

\item
Better results are obtained by testing from the beginning onwards;

\item
Tests need not be rebuilt from scratch to localise later bugs;

\end{itemize}

%%....................................................................
\subsection{External Interface Tests}

The external interface tests are mandatory.
For at least one class \code{MyClass} in a given hierarchy there should be
an executable \file{test-MyClass}.
It should send output to \code{cout}, which is read by a report script.
The only {\em required} output is a line
\begin{verbatim}
  n tests failed.
\end{verbatim}
when $n > 0$ tests fail.

In general, for every public method in the hierarchy,
\file{test-MyClass} should output a line of the form

\code{[+-] <function prototype>}

\noindent where \code{+} or \code{-} indicates whether the tests for this
method were successful. Additional tests which do not focus on a
particular method may be appropriate; see, e.g., \\
\file{magnus/back\_end/general/test/test-List.C}.

Note that for container classes such as \code{ListOf} and
\code{SetOf}, it is straightforward to test each method in isolation,
but in the case of non-trivial group theoretic code this may not be
feasible. Here are some alternative techniques we have employed:

\begin{itemize}
\item
For Knuth-Bendix, we have a fixed suite of test inputs along with the
expected output; the latter can be mechanically compared to the actual
output.

\item
For the one relator word problem, we have a program which generates
relators and trivial words at random. Since the expected output is
known regardless of the input, the actual output is easy to test
mechanically for correctness. The parameters of the random generation
are fixed at compile time, so we can repeat the test.
\end{itemize}


Other conventions:

\begin{itemize}
\item
The tests must be defined exclusively in terms of the
public interfaces of the classes in the hierarchy.

\item
If the test requires input data from a file, this data must
be in a file
\file{test-MyClass.data}

\item
If the test requires another program to generate data, the
external program must of course also be provided, and called from
\file{test-MyClass} with a system call. If some data is
generated randomly, the random generator must be seeded the same way for
each invocation of \file{test-MyClass}.

\item
If the test produces output on \code{cout} which should be compared with
a file of correct output, the latter should be called
\file{test-MyClass.mastertestout}.
If the output needs to be evaluated by an external program, then
of course this program should also be provided, and called from
\file{test-MyClass} with a system call.

\end{itemize}

%%....................................................................
\subsection{Low Level Tests} % @rn\label{low_level_tests}

%% . . . . . . . . . . . . . . . . . . . . . . . . . . . . . . . . . .
\subsubsection{General Principle}

In the course of implementation, you may find yourself creating ad hoc
low-level tests as you go along to debug or to test the correct
low-level functioning of your implementation.  Typically, this may
involve adding special, not-for-the-final-implementation friend and
(static) member functions with access to the private parts of a class
or several classes.  These should be considered part of the
implementation, and should be conditionally compiled with the flag
\code{DEBUG}, defined in \file{magnus/back\_end/global/global.h}, set.

%% . . . . . . . . . . . . . . . . . . . . . . . . . . . . . . . . . .
\subsubsection{Procedure}

Instead of `throwing away' such effort after use,
try to keep the functions and programs for reuse.

\begin{itemize}
\item
Special friend and member
functions should be made part of the permanent code but
conditionally compilable
dependent on the compiler toggle \code{DEBUG} being defined.

\item
Instead of writing \code{main()} functions, give these global
functions another name, and keep them in file
\file{debug-MyClass.C}

\item
Data goes into files
\file{debug-MyClass.[...]}

\end{itemize}

%% . . . . . . . . . . . . . . . . . . . . . . . . . . . . . . . . . .
\subsubsection{Documentation}

As these tests are supposed be reusable, it is essential for them to
be sufficiently documented, through a description and adequate
comments in the testing code.  The description is in an appropriate
subsection of the Reference Manual.


%% This is AMSLaTeX source, included by guidelines_incl.tex.


%%--------------------------------------------------------------------
\section{A {\bf cvs} Tutorial}

%%....................................................................
\subsection{Introduction}


The \magnus\ project uses the Concurrent Versions System {\bf cvs(1)}
for version control.

This tutorial is not a substitute for the {\bf man} pages. Rather, it
is an overall explanation of the {\bf cvs} commands, and a discussion
of the very idea of version control.

{\bf cvs} keeps a unique directory tree, called the {\em master
repository}, which contains {\bf rcs(1)} archives of all project
files. Each developer keeps at least one {\em working directory}, in
which to perform the usual edit-compile-link-run-crash cycles. The
directory structure of a working directory usually contains a
subforest of the master tree.

The {\bf cvs} paradigm differs from those with which you may be
familiar, as it permits concurrent editing of files. That is, it does
not lock a given source file so that only one developer at a time can
edit it.

A locking mechanism attempts to prevent conflicting changes to a
source file by two or more people {\em before the fact}, by announcing
to all developers that the file is `locked' by someone, and therefore
should not be edited by anyone else.  However, any such mechanism has
loopholes through which it can be inadvertently defeated. This can
result in one revision shadowing another with no warning.

In recognition of the fact that conflicts will occur, {\bf cvs}
automatically checks for them, and attempts to resolve them, {\em
after the fact}. This ensures that conflicts are always caught, but
places a greater burden on those developers who edit a file
concurrently. Because, as they attempt to check their changes into the
master, {\bf cvs} will run {\bf merge(1)} on the original revision and
the two changed ones. If the changes do not overlap, they are merged
automatically, which is presumably the right thing.  Otherwise, one of
the developers will have to resolve the conflict by hand, using the
output from {\bf merge}.


%%....................................................................
\subsection{Getting Started}

You must set the environment variable CVSROOT to the full path of the
master repository. For example, put the line

\vspace{8pt}

{\tt setenv CVSROOT /home/users/magnus\_repository}

\vspace{8pt}

\noindent in your {\tt .cshrc} file.

\vspace{8pt}

\begin{quotation}

The contents of \$CVSROOT are to be manipulated by {\bf cvs} only.

NO ONE, UNDER ANY CIRCUMSTANCES OF ANY KIND WHATSOEVER, IS TO DIRECTLY
ALTER OR REMOVE ANY FILE OR DIRECTORY IN OR BELOW \$CVSROOT.

\end{quotation}


You must make a working directory. You do not necessarily need a copy
of the entire master, but this is the easiest to do at first.  Switch
to the directory where you want your working directory to go, then
invoke the command

\vspace{8pt}

{\tt cvs checkout magnus}

\vspace{8pt}

This command is decribed in more detail, below. Whatever you do,
remember that you must work in a directory tree created by the {\bf
checkout} command, since {\bf cvs} will not recognize a tree which it
did not create, unless you take special action (see the {\bf import}
command).


%%....................................................................
\subsection{Common Commands}

There are only five {\bf cvs} commands which a developer typically
needs. Perhaps it goes without saying, but one should not use other
commands, or the more esoteric options of the five below, until one
fully understands the consequences.

You will see that there is some subtle interaction between the {\bf
checkout}, {\bf add}, and {\bf update} commands, because {\bf cvs}
must be explicitly told which files in your working directory are to
be considered for inclusion in the master. This conveniently allows
you to have completely private files in your working directory,
without {\bf cvs} constantly trying to inappropriately check them into
the master.

Below, `{\em module}' means a file or directory path relative to the
master root, or a symbolic tag which refers to a collection of files.
And `{\em files}' means files or directories in the current directory.

\begin{description}

\item{{\bf cvs checkout} [options] {\em modules}...}

The purpose of this command is to make a working copy of some
subforest of the master, along with the administrative files which
{\bf cvs} needs to operate.

This is typically used to initialize your working directory, but you
can also use it to update an existing working subdirectory (see {\bf
cvs(1)}).

You may also, for example, use this command to create a duplicate of
some working subdirectory, in which to do experimental work without
disturbing the `real' one.

For this reason, the {\bf create} command does not automatically put
the checked out files in the same location, relative to the master
root, as they are in the master. Rather, it puts the roots of any
trees it creates in the directory from which it was invoked.

If any of the {\em modules} you specify are directories, {\bf
checkout} is recursive, unless you give the {\tt -l} (for local)
option. The {\tt -n} option prints a summary of what would happen if
the command were executed, but does not actually change anything. See
{\bf cvs(1)} for more options.



\item{{\bf cvs update} [options] {\em files}...}

This changes your working directory, not the master repository.

This brings your working files up to date with any changes which
others have made in the master since the last time you did {\bf
update} or {\bf checkout}. It is recursive unless you give the {\tt
-l} option. It {\em does not} check any working files you have changed
into the master. Use {\bf commit} for that.

Examples:

\begin{itemize}

\item
Someone has fixed a bug in {\tt
magnus/back\_end/general/include/List.h}, and now you want it.
Switch to the directory which contains your working root, and do

\vspace{8pt}

{\tt cvs update magnus/back\_end/general/include/List.h}

\vspace{8pt}

\item
As above, but you can't remember what file in {\tt general/} was
fixed.  If you do

\vspace{8pt}

{\tt cvs update magnus/back\_end/general}

\vspace{8pt}

\noindent then all of your files in {\tt general/} are brought up to date.

\vspace{8pt}

But what if you are working on, say, {\tt general/src/Set.C}, and you
are not ready to check it in? Not to worry. The {\bf update} command
never checks files in.

\vspace{8pt}

But what if {\em someone else} has also edited {\tt Set.C} in the mean
time? In that case, {\bf update} will try to merge the other changes
with your working copy of {\tt Set.C}. If you were working on disjoint
parts of {\tt Set.C}, the merge is automatic. If you were both editing
the same lines, {\bf update} will inform you that there is a conflict
which you must resolve by hand (see \S\ref{resolving_conflicts}).

\end{itemize}



\item{{\bf cvs add} [options] {\em files}...}

As you add new files to your working tree which are to be included in
the master, you must inform {\bf cvs} of them. Otherwise, {\bf cvs}
will assume that the file is for your private use, not part of the
project, and ignore it.

After you have made a new file, e.g., {\tt magnus/doc2/blurb.tex}, tell
{\bf cvs} that it is part of the project by switching to {\tt
magnus/doc2/}, and doing

\vspace{8pt}

{\tt cvs add blurb.tex}

\vspace{8pt}

Of course this assumes that the directory {\tt magnus/doc2/} was
either created by the {\bf checkout} command, or that {\bf cvs} was
told about it with the {\bf add} command.

The file `{\tt blurb.tex}' will not be checked into the master until
your run {\bf commit}.



\item{{\bf cvs remove} [options] {\em files}...}

Use this command to announce that the indicated files and directories
are no longer part of the project. The command is recursive unless you
give the {\tt -l} option.

Consider carefully whether it is for you to decide that a file no
longer part of the project.

The command requires that you first {\tt rm} the indicated files and
directories from your working tree. It does not change the master
until you run {\bf commit}.

Do {\em NOT\/} use this command when you merely wish to remove part of
your working tree, pending a {\bf checkout} to restore it. See the
{\bf release} command.



\item{{\bf cvs commit} [options] [{\em files}...]}

Use this command to check files into the master. They must all reside
in directories created with the {\bf checkout} command, or announced
with the {\bf add} command.

If you do not specify a file or directory, the current directory is
assumed. This command is recursive unless you give the {\tt -l}
option. It supports the {\tt -n} option, which prints a summary of
what would happen if you did commit, without actually changing
anything.

{\bf cvs} may detect that some of the specified files are not up to
date with the current revision in the master. That is, someone else
has also edited one of the files you want to commit, since the last
time you ran {\bf commit} or {\bf update}. In this case, {\bf cvs}
issues a warning to run {\bf update}, and exits without commiting any
files.

By default, this will invoke the editor pointed to by your EDITOR
environment variable, to ask for a log message. You can suppress this
with the {\tt -m ""} option. You may wish to alias this.


\end{description}



%%....................................................................
\subsection{Other Commands}


\begin{description}


\item{{\bf cvs diff} [options] [{\em files or directories}...]}



\item{{\bf cvs export} [options] [{\em files or directories}...]}

\item{{\bf cvs import} [options] [{\em files or directories}...]}

\item{{\bf cvs release} [options] [{\em files or directories}...]}

\item{{\bf cvs status} [options] [{\em files or directories}...]}


\end{description}



%%....................................................................
\subsection{Resolving Conflicts}\label{resolving_conflicts}



%%--------------------------------------------------------------------
\section{Templates and \code{gcc 2.6.3}}

The usual hack to work around a broken template implementation (e.g.
any version of \code{gcc}) is to make every single member function of
the template class inline, and just live with the massive overhead.
But this is not always possible; for example, class \code{SetOf}
cannot be completely inlined.

\code{gcc 2.6.3} doesn't have a proper template repository, so we face
the following problem: suppose you have two source files, \file{foo.C}
and \file{bar.C}, each of which make the same instantiation of a
template class, say, \code{Baz<char>}. In what object file does poor
\code{gcc} put the member functions for \code{Baz<char>}? Both
\file{foo.o} and \file{bar.o}? Then the linker complains of multiply
defined symbols (justifiably), so that's not what \code{gcc} does. Instead,
the `repository' for the member functions of the template instance is,
by convention, the object file corresponding to the source file for
the template class. How does \code{gcc} know which instances to put in
the `repository'? It doesn't. {\em You have to explicitly instantiate
every conceivable instance of the template in the source file for the
template!}

What this means for us is that if you want to use, say,
\code{SetOf<Word>} {\em anywhere}, then \file{Set.C} must contain
the lines

\begin{verbatim}
#include "Word.h"
template class SetOf<Word>;
\end{verbatim}

But that's not all. How does \file{Set.C} know where to find
\file{Word.h}? The makefiles as of 3/1/95 put include search paths in
one-to-one correspondence with library dependencies. As it happens,
\file{libgeneral} (where \file{Set.o} is) depends on \file{libElt} (where
\file{Word.o} is), so there is no problem with this example.

Now suppose that some file in \file{FSA/} wants to use a
\code{SetOf<StatePair>}, where class \code{StatePair} is defined somewhere
in \file{FSA/}, as you might expect. Then \file{Set.C} must contain
the lines

\begin{verbatim}
#include "StatePair.h"
template class SetOf<StatePair>;
\end{verbatim}

\noindent
But then \file{Set.C} must be compiled with include path containing
\file{FSA/include/}, but without \file{libgeneral} depending on
\file{libFSA}. The makefiles don't allow for this yet, and won't be
easy to fix.

A quick kludge might be to hard-wire the path, so the above would be:

\begin{verbatim}
#include MAGNUS_HOME "/back_end/FSA/include/StatePair.h"
\end{verbatim}

\noindent
but this must be done for any nested inclusions too.




%%--------------------------------------------------------------------
\section{Writing User Classes}\label{user_classes}

%%....................................................................
\subsection{Introduction}

In C++, a class usually contains data members along with methods
which operate on the data. In our scheme, we put the data for a user
class in a separate data representation class, and our user classes
contain (by derivation) precisely one data member: a private pointer to a data
representation object. Moreover, our user classes have no virtual methods;
overridable methods are delegated to the data representation.

Because objects in user classes occupy only as much memory as a
pointer, they can be efficiently passed as parameters, copied,
assigned, etc.  These operations involve manipulating a reference
count in the data representation, the cost of which is negligible.
The benefits are

\begin{itemize}
\item
We use a single mode: automatic variables and parameters to public
methods are always objects, rather than explicit pointers to objects.

\item
Memory is automatically protected and deallocated.

\item
User objects can be copied and assigned without explicit casts.

\item
The data representation is hidden, so the user has no choice
but to access it safely through the public interface.

\item
The data representation can change without changes higher up.

\item
Important low-level code, such as that for reference counting, need only
be written once.

\end{itemize}

%%....................................................................
\subsection{Data Hiding}

We discuss in general how to convert the `usual' sort of class
hierarchy to our scheme.

Let \code{H} be a class hierarchy in which there is at least one
class, \code{Root}, which does not derive from a class in \code{H},
and such that every class in \code{H} with no subclasses derives from
\code{Root}. Each class in \code{H} may contain data members and/or
(pure) virtual methods, and derivation may be virtual.

We build a user object hierarchy \code{HObj} and a data representation
hierarchy \code{HRep} from \code{H}. It may be helpful to think of the
case where \code{H} consists of an abstract root with two concrete
subclasses:

\lskip
\begin{verbatim}
  class Root;            // abstract
  class X : public Root; // concrete
  class Y : public Root; // concrete
\end{verbatim}
\lskip

The corresponding class diagrams are:

%\lskip
%\begin{verbatim}
%          RefCounter            ObjectOf<RootRep>
%              |                        |
%  XData    RootRep    YData         RootObj
%       \  /       \  /             /       \
%       XRep       YRep           XObj     YObj
%\end{verbatim}
%\lskip

\vspace{20pt}

\epsfbox{ps/ObjectOf.ps}

\vspace{20pt}


Corresponding to \code{Root}, \code{HRep} contains a class
\code{RootRep} which derives directly from \code{RefCounter}.
\code{HObj} contains a class \code{RootObj} which derives directly
from \code{ObjectOf<RootRep>}.  Thus the compiler automatically makes
a copy constructor, a destructor, and an assignment operator for
\code{RootObj} from those of \code{ObjectOf<RootRep>}.  Subclasses of
\code{RootObj} should not define these.  \code{ObjectOf<RootRep>} has
a pointer to a \code{RootRep} which is {\em private}, so
\code{RootObj} can't use it directly. However, \code{RootObj} inherits
protected methods

\lskip
\begin{verbatim}
  const RootRep *look() const;
  RootRep *change();
  RootRep *enhance();
\end{verbatim}
\lskip

\noindent from \code{ObjectOf<RootRep>}. The \code{look()} method simply
returns the pointer; since it is \code{const}, only \code{const}
methods of \code{RootRep} can be called through it, so no cloning is
necessary to protect the data.  The \code{change()} method ensures
that \code{RootRep} clones its data if necessary before returning the
pointer.  The \code{*enhance()} method is a `non-const' version of
\code{look}, to be used with extreme caution to alter a \code{Rep} in
a semantically constant way.  \code{RootObj} and its derivatives can
use these to safely look at and change the data in a (subclass of)
\code{RootRep}, via the latter's public interface. \code{RootRep} must
declare

\begin{verbatim}
  virtual RootRep *clone( ) const = 0;
\end{verbatim}

\noindent the instantiations in concrete subclasses do a deep copy.

Now let \code{C} be a class in \code{H}. \code{HObj} contains a
corresponding class \code{CObj} with no data members.  If \code{C} has
any data members, then \code{HRep} contains a class \code{CData} in
which the data is put.  If \code{C} has any virtual methods, then
\code{HRep} contains a class \code{CRep} in which {\em only} these
virtual methods are defined, and \code{CObj} has corresponding
non-virtual methods which delegate to the ones in \code{CRep}. If
\code{CData} exists then \code{CRep} derives directly from it.

Except for the \code{CRep}/\code{CData} duality, the derivation structures of
\code{HObj} and \code{HRep} both match that of \code{H}.

Now we turn to what must and must not be in each of \code{CObj},
\code{CRep}, and \code{CData}.

%% . . . . . . . . . . . . . . . . . . . . . . . . . . . . . . . . . .
\subsubsection{Object classes}

A class \code{CObj} contains no data members (besides the inherited
one) and no virtual methods. Any method which is common to all \code{
C}'s, i.e. a method which one would want to declare virtual in \code{
CObj}, will instead {\em delegate} to a corresponding virtual method
in \code{CRep}. We call these {\em pseudo-virtual} methods. Note that
pseudo-virtual methods are not overridden in \code{HObj} -- their
delegates are overridden in \code{HRep}. Two examples of delegation:

\lskip
\begin{verbatim}
   virtual CRep *CRep::f(...) const;
   CObj CObj::f(...) const { return CObj( look()->f(...) ); }
\end{verbatim}
\lskip
\begin{verbatim}
   virtual int CRep::g(...);
   int CObj::g(...) { return change()->g(...); }
\end{verbatim}
\lskip

Any non-virtual methods in \code{C} go in \code{CObj}, and are defined
in terms of the public interface(s) of \code{CData} and/or \code{CRep}
via \code{look()} and \code{change()}. Unless \code{CObj} is \code{
RootObj}, it must shadow the inherited \code{look()}, \code{change()}
and \code{enhance()} methods so that they return a \code{CRep*}.
Since \code{look()} and \code{change()} are protected, \code{CObj}
must provide explicit data access methods to the user.

The names of the classes in \code{HObj} may as well match those in
\code{H}; we use the \code{Obj} suffix here to avoid confusion.

\code{CObj} may need to provide a cast constructor which takes a
{\em superclass}. This makes sense because the only data members in
\code{CObj} and its superclass is a (castable) pointer. For example,

\lskip
\begin{verbatim}
class Word : public Elt {
public:  
  Word( const Elt& e ) : Elt( e ) { }
}
\end{verbatim}
\lskip

%% . . . . . . . . . . . . . . . . . . . . . . . . . . . . . . . . . .
\subsubsection{Representation classes}

The \code{CRep} class contains any data members which do not belong in
\code{CData}. The only methods it contains are initialization
constructors as appropriate, and virtual methods which mirror any
pseudo-virtual ones in \code{CObj}.  These are either instantiated or
overridden in a subclass of \code{CRep}. Methods which construct a new
\code{CRep} should return a \code{CRep*}. This is wrapped by
\code{CObj} using the \code{ObjectOf<RootRep>} constructor which takes
a \code{RootRep*}.

The real purpose of \code{CRep} is to avoid re-coding the virtual
methods of \code{C} in each \code{CData} class, as we may want several
implementations of \code{CData} for a given \code{C}.  The virtual
methods can be written once in terms of \code{CData}'s (fixed) public
interface.\footnote{A virtual method might be too inefficient this
way, and have to be pushed up to \code{CData}.} When the virtual
methods are simple (or there are none), the \code{CRep} and
\code{CData} classes may be amalgamated into one.

%% . . . . . . . . . . . . . . . . . . . . . . . . . . . . . . . . . .
\subsubsection{Data classes}

The \code{CData} class contains and maintains {\em all} of the data for
a \code{C}. Its public interface provides only access methods, a copy
constructor which does deep copy, other constructors as appropriate,
and a destructor. The data members are private; it must be impossible
for any function besides a \code{CData} method to assume anything
about the structure of the data beyond what is indicated in the public
interface.
